\documentclass[11pt]{article}
\usepackage{graphicx}
\usepackage{amssymb}

\title{MATLAB Milou Toolbox}
\author{Kristoffer Andersson}
\begin{document}
\maketitle

\section{ArrayMatrix}
ArrayMatrix holds an array of square matrices (typical a suite of measured S-parameters). The class can perform various operations on this array.

\subsection*{arraymatrix}
M = arraymatrix
returns an empty structure with matrix dimension 2

M = arraymatrix(A), where A is a matrix of size n*n*m.
returns a structure containingA with matrix dimension n and length m.

M = arraymatrix(A,m), where A is a matrix of size n*n and m is a positive scalar.
returns an structure with A repeated m times.

M = arraymatrix(n,m), where n and m is positive scalars.
returns a structure containing zeros with matrix dimension n and length m.

\subsection*{display}
Displays the matrix dimension and number of elements in a arraymatrix-object.

\subsection*{inv}
X = inv(M), where M is a arraymatrix.
returns the inverse of matrix M. If dimension of M is equal to two a explicit algorithm is used, otherwise for-loops are used.

\subsection*{mtimes}
Overloads the multiplication operator *. This routine handles multiplication with scalars, square matrices and other ArrayMatrix-objects. The routine uses for-loops, could possibly be rewritten to use explicit algorithm when n=2.

\subsection*{rdivide}
Overloads the division operator /. This routine handles division with scalars, square matrices and other ArrayMatrix-objects. The routine uses for-loops, could possibly be rewritten to use explicit algorithm when n=2.

\subsection*{plus}
Overloadsthe addition operator -. This routine handles addition with scalars, square matrices and other ArrayMatrix-objects. 

\subsection*{minus}
Overloads the subtraction operator -. This routine handles subtraction with scalars, square matrices and other ArrayMatrix-objects. 

\subsection*{uminus}
Overloads the unary minus operator.

\subsection*{subsref}
Overloads the index operator (). Two formats are supported: x=A(m) or x=A(x,y). In the former case x returns the m'th object in the array in the latter x returns the (x,y)-element in the array.

\subsection*{subsasgn}
Overloads the assigment index operator (). Operates almost like the inverse of the above. The constructor is run on the right-hand-side thus enabling a wide range of input. Calling the method with A(m)=B. Sets the m'th element equal to the matrix B. Calling with A(x,y) = b, sets each (x,y)-element in A equal to the vector b.

\subsection*{length}
Returns the number of elements in the ArrayMatrix.

\subsection*{get}
Provides access to the internals of the class. x=get(A,property). Valid properties are 'n','m' and 'mtrx'.

\subsection*{conj}
Calculates the conjugate of each matrix in the array.

\subsection*{real}
Calculates the real of each matrix in the array.

\subsection*{imag}
Calculates the imaginary of each matrix in the array.

\subsection*{abs}
Calculates the absolute value of each matrix-value in the array.

\subsection*{angle}
Calculates the angle (in radians) of each matrix-value in the array.

\subsection*{det}
Calculates the determinant of each matrix in the array.

\subsection*{ident}
Returns an appropriate identity matrix. Quite useless provided for backwards compatibility.

\subsection*{skip}
Skips (or drops) the selected row and column in each matrix. C = skip(A,[x y]) drops the x'th row and y'th column in each matrix.


 \end{document}