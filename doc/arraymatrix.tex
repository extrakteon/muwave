\chapter{ArrayMatrix}

\section{arraymatrix}\hypertarget{arraymatrix}
Class constructor.
\subsection{Syntax}
\begin{verbatim}
    M = arraymatrix
    M = arraymatrix(A)
    M = arraymatrix(A,m)
    M = arraymatrix(nx,ny,m)
\end{verbatim}

\subsection{Description}
\verb"M = arraymatrix" returns an empty structure with element
matrix dimension 2*2.\\
\verb"M = arraymatrix(A)" where A is a matrix of size nx*ny*m.
Returns a structure containing A with matrix dimension nx*ny and
length m.\\
\verb"M = arraymatrix(A,m)" where A is a matrix of size nx*ny and
m is a positive scalar. returns a structure with matrix A repeated
m times.\\
\verb"M = arraymatrix(nx,ny,m)" where nx, ny and m is positive
scalars. returns a structure containing zeros with matrix
dimension nx*ny and length m.
\subsection{Examples}
\begin{verbatim}
    >> A=[1,2;3,4];
    >> arraymatrix(A)
    arraymatrix-object
     dimension:  2 x 2
     elements:   1

    >> arraymatrix(A,3)
    arraymatrix-object
     dimension:  2 x 2
     elements:   3
\end{verbatim}

%%%%%%%%%%%%%%%%%%%%%%%%%
\vspace{3mm} \hrule
\section{display}\hypertarget{display}
Display arraymatrix.
\subsection{Syntax}
\begin{verbatim}
    display(M)
\end{verbatim}
\subsection{Description}
Overloads \verb"display". Displays info about the arraymatrix when
semicolon isn't used for line termination.

%%%%%%%%%%%%%%%%%%%%%%%%%
\vspace{3mm} \hrule
\section{inv}\hypertarget{inv}
Inverse of each element in arraymatrix.
\subsection{Syntax}
\begin{verbatim}
    Y = inv(X)
\end{verbatim}
\subsection{Description}
\verb"Y = inv(X)" returns the inverse of the square arraymatrix
\verb"X". If dimension of \verb"X" is equal to two an explicit
algorithm is used, otherwise for-loops are used.

%%%%%%%%%%%%%%%%%%%%%%%%%
\vspace{3mm} \hrule
\section{mtimes}\hypertarget{mtimes}
Matrix multiply.
\subsection{Syntax}
\begin{verbatim}
    C = A*B
    C = A*b
    C = a*B
\end{verbatim}
\subsection{Description}
Overloads the multiplication operator \verb"*". This routine
handles multiplication with scalars, matrices and other
ArrayMatrix-objects.
\subsection{See Also}
\hyperlink{rdivide}{rdivide} \hyperlink{plus}{plus}
\hyperlink{minus}{minus} \hyperlink{uminus}{uminus}

%%%%%%%%%%%%%%%%%%%%%%%%%
\vspace{3mm} \hrule
\section{rdivide}\hypertarget{rdivide}
Right matrix divide
\subsection{Syntax}
\begin{verbatim}
    C = A/B
    C = A/b
    C = a/B
\end{verbatim}
\subsection{Description}
Overloads the division operator \verb"/". This routine handles
division with scalars, square matrices and other
ArrayMatrix-objects.
\subsection{See Also}
\hyperlink{mtimes}{mtimes} \hyperlink{plus}{plus}
\hyperlink{minus}{minus} \hyperlink{uminus}{uminus}

%%%%%%%%%%%%%%%%%%%%%%%%%
\vspace{3mm} \hrule
\section{plus}\hypertarget{plus}
Addition
\subsection{Syntax}
\begin{verbatim}
    C = A + B
    C = A + b
    C = a + B
\end{verbatim}
\subsection{Description}
Overloads the addition operator \verb"-". This routine handles
addition with scalars, matrices and other ArrayMatrix-objects.
\subsection{See Also}
\hyperlink{mtimes}{mtimes} \hyperlink{rdivide}{rdivide}
\hyperlink{minus}{minus} \hyperlink{uminus}{uminus}

%%%%%%%%%%%%%%%%%%%%%%%%%
\vspace{3mm} \hrule
\section{minus}\hypertarget{minus}
Subtraction.
\subsection{Syntax}
\begin{verbatim}
    C = A-B
    C = A-b
    C = a-B
\end{verbatim}
\subsection{Description}
Overloads the subtraction operator \verb"-". This routine handles
subtraction with scalars, matrices and other ArrayMatrix-objects.
\subsection{See Also}
\hyperlink{mtimes}{mtimes} \hyperlink{rdivide}{rdivide}
\hyperlink{plus}{plus} \hyperlink{uminus}{uminus}

%%%%%%%%%%%%%%%%%%%%%%%%%
\vspace{3mm} \hrule
\section{minus}\hypertarget{minus}
Unary minus.
\subsection{Syntax}
\begin{verbatim}
    Y = -Y
\end{verbatim}
\subsection{Description}
\subsection*{uminus}
Overloads the unary minus operator.
\subsection{See Also}
\hyperlink{mtimes}{mtimes} \hyperlink{rdivide}{rdivide}
\hyperlink{plus}{plus} \hyperlink{minus}{minus}

%%%%%%%%%%%%%%%%%%%%%%%%%
\vspace{3mm} \hrule
\section{transpose}\hypertarget{transpose}
Matrix transpose.
\subsection{Syntax}
\begin{verbatim}
    Y = X.'
\end{verbatim}
\subsection{Description}
Overloads the transpose operator \verb".*".
\subsection{See Also}
\hyperlink{ctranspose}{ctranspose}

%%%%%%%%%%%%%%%%%%%%%%%%%
\vspace{3mm} \hrule
\section{ctranspose}\hypertarget{ctranspose}
Hermitian matrix transpose.
\subsection{Syntax}
\begin{verbatim}
    Y = X'
\end{verbatim}
\subsection{Description}
Overloads the hermitian transpose operator \verb"*".
\subsection{See Also}
\hyperlink{transpose}{transpose}

%%%%%%%%%%%%%%%%%%%%%%%%%
\vspace{3mm} \hrule
\section{subsref}\hypertarget{subsref}
Subscripted reference.
\subsection{Syntax}
\begin{verbatim}
    Y = X(I)
    Y = X(nx,ny)
\end{verbatim}
\subsection{Description}
\verb"Y=X(I)" is an arraymatrix formed from the elements of
\verb"X" specified by the index vector \verb"I".\\
\verb"Y=X(nx,ny)" is a vector containing the \verb"(nx,ny)"
element of each matrix.
\subsection{Se Also}
%\hyperref{subsasgn}{subsasgn}

%%%%%%%%%%%%%%%%%%%%%%%%%
\vspace{3mm} \hrule
\section{subsasgn}\hypertarget{subsasgn}
Subscripted assigment.
\subsection{Syntax}
\begin{verbatim}
    X(I) = Y
    X(nx,ny) = Y
\end{verbatim}
\subsection{Description}
\verb"X(I)=Y" assigns the values of \verb"Y" to the positions of
\verb"X" specified by the index vector \verb"I".\\
\verb"X(nx,ny)=Y" assigns the values of vector \verb"Y" to the
\verb"(nx,ny)" element of each matrix.
\subsection{Se Also}
%\hyperref{subsref}{subsref}

%%%%%%%%%%%%%%%%%%%%%%%%%
\vspace{3mm} \hrule
\section{length}\hypertarget{length}
Length of object.
\subsection{Syntax}
\begin{verbatim}
    Y = length(X)
\end{verbatim}
\subsection{Description}
\verb"Y = length(X)" is the number of elements in X.
\subsection{See Also}
%\hyperref{size}{size}

%%%%%%%%%%%%%%%%%%%%%%%%%
\vspace{3mm} \hrule
\section{size}\hypertarget{size}
Size of object.
\subsection{Syntax}
\begin{verbatim}
    [NX,NY,M] = size(X)
    Y = size(X,dim)
\end{verbatim}
\subsection{Description}
\verb"[NX,NY,M] = size(X)" is the vector containing the size of
\verb"X".\\
\verb"Y = size(X,dim)" is the size of \verb"X" along the dimension
\verb"dim".
\subsection{See Also}
%\hyperref{length}{length}

%%%%%%%%%%%%%%%%%%%%%%%%%
\vspace{3mm} \hrule
\section{get}\hypertarget{get}
Get object properties.
\subsection{Syntax}
\begin{verbatim}
    Y = get(X,property)
\end{verbatim}
\subsection{Description}
Provides access to the internals of the class.\\
\verb"Y = get(X,'nx')" is the number of rows in each matrix.\\
\verb"Y = get(X,'ny')" is the number of columns in each matrix.\\
\verb"Y = get(X,'m')" is the number of elements in the array.\\
\verb"Y = get(X,'mtrx')" is the three dimensional matrix holding
the data.

%%%%%%%%%%%%%%%%%%%%%%%%%
\vspace{3mm} \hrule
\section{conj}\hypertarget{conj}
Matrix conjugate.
\subsection{Syntax}
\begin{verbatim}
    Y = conj(X)
\end{verbatim}
\subsection{Description}
Calculates the conjugate of each matrix in the array.

%%%%%%%%%%%%%%%%%%%%%%%%%
\vspace{3mm} \hrule
\section{real}\hypertarget{real}
Real part.
\subsection{Syntax}
\begin{verbatim}
    Y = real(X)
\end{verbatim}
\subsection{Description}
Calculates the real part of each matrix in the array.

%%%%%%%%%%%%%%%%%%%%%%%%%
\vspace{3mm} \hrule
\section{imag}\hypertarget{imag}
Imaginary part.
\subsection{Syntax}
\begin{verbatim}
    Y = imag(X)
\end{verbatim}
\subsection{Description}
Calculates the imaginary part of each matrix in the array.

%%%%%%%%%%%%%%%%%%%%%%%%%
\vspace{3mm} \hrule
\section{angle}\hypertarget{angle}
Angle of complex number.
\subsection{Syntax}
\begin{verbatim}
    Y = angle(X)
\end{verbatim}
\subsection{Description}
Calculates the angle (in radians) of each matrix-entry in the
array.

%%%%%%%%%%%%%%%%%%%%%%%%%
\vspace{3mm} \hrule
\section{abs}\hypertarget{abs}
Absolute value.
\subsection{Syntax}
\begin{verbatim}
    Y = abs(X)
\end{verbatim}
\subsection{Description}
Calculates the absolute value of each matrix-entry in the array.

%%%%%%%%%%%%%%%%%%%%%%%%%
\vspace{3mm} \hrule
\section{det}\hypertarget{det}
Determinant.
\subsection{Syntax}
\begin{verbatim}
    Y = det(X)
\end{verbatim}
\subsection{Description}
Calculates the determinant of each matrix in the array.

%%%%%%%%%%%%%%%%%%%%%%%%%
\vspace{3mm} \hrule
\section{ident}\hypertarget{ident}
Determinant.
\subsection{Syntax}
\begin{verbatim}
    Y = ident(X)
\end{verbatim}
\subsection{Description}
\subsection*{ident}
Returns an appropriate identity matrix. Quite useless provided for
backwards compatibility.

%%%%%%%%%%%%%%%%%%%%%%%%%
\vspace{3mm} \hrule
\section{skip}\hypertarget{skip}
Drops rows and columns.
\subsection{Syntax}
\begin{verbatim}
    Y = skip(X,[row col])
\end{verbatim}
\subsection{Description}
\subsection*{ident}
Skips (or drops) the selected row and column in each matrix.
\verb"Y = skip(X,[row col])" drops the \verb"row"'th row and
\verb"col"'th column in each matrix.
