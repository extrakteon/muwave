\chapter{Mna}
Simple tool for calculating response of a linear circuit using
Modified Nodal Analysis.
\section{mna}\hypertarget{mna}
Class constructor.
\subsection{Syntax}
\begin{verbatim}
    Y = mna
    Y = mna(n)
\end{verbatim}

\subsection{Description}
\verb"Y = mna" returns a default mna object with 1 node.\\
\verb"Y = mna(n)", where \verb"n" is a positive scalar, returns a mna object with \verb"n" nodes.\\

%%%%%%%%%%%%%%%%%%%%%%%%%
\vspace{3mm} \hrule
\section{stamp}\hypertarget{stamp}
Inserts the MNA footprint of a circuit element into the
mna-object.
\subsection{Syntax}
\begin{verbatim}
    Y = stamp(Y,type,element,conn)
\end{verbatim}
\subsection{Description}
\verb"Y = stamp(Y,type,element,conn)" returns a mna-object with
the circuit element \verb"type" with name \verb"element" between
nodes \verb"conn".

\verb"type" can be a reciprocal two port element
\verb"'R','G','L','C'" or a non reciprocal four port
\verb"'VCCS','GY'". \verb"element" can be any string describing
the element name. For two ports \verb"conn" is a two element
vector containing the nodes of the circuit element, for four ports
it is a four element vector.

\subsection{Example}
Produces a indefinite MNA description of a standard transistor
$\pi$-pad.
\begin{verbatim}
    >> Y = stamp(mna(3),'C','Cpg',[1 3]);
    Y = stamp(Y,'L','Lg',[1 2]);
    Y = stamp(Y,'C','Cpg',[2 3]);
    >> Y
    ans =
        '+s.*Cpg+1./(s*Lg)'    '-1./(s*Lg)'           '-s.*Cpg'
        '-1./(s*Lg)'           '+1./(s*Lg)+s.*Cpg'    '-s.*Cpg'
        '-s.*Cpg'              '-s.*Cpg'              '+s.*Cpg+s.*Cpg'
    ans =
        'Cpg'    'Lg'
\end{verbatim}

Example of a voltage controlled current source.
\begin{verbatim}
    >> Y=stamp(mna(4),'VCCS','gm',[1 2 3 4])
    ans =
           []     []       []     []
        '+gm'     []    '-gm'     []
           []     []       []     []
        '-gm'     []    '+gm'     []
    ans =
        'gm'
\end{verbatim}

Example of a gyrator.
\begin{verbatim}
    >> Y=stamp(mna(4),'GY','g',[1 2 3 4]) ans =
      []    '+g'      []    '-g'
    '-g'      []    '+g'      []
      []    '-g'      []    '+g'
    '+g'      []    '-g'      []
    ans =
        'g'
\end{verbatim}

%%%%%%%%%%%%%%%%%%%%%%%%%

\vspace{3mm} \hrule
\section{freq}\hypertarget{freq}
Sets the frequencies for subsequent calculation of a numeric
MNA-matrix.
\subsection{Syntax}
\begin{verbatim}
    Y = freq(Y,frequencies)
\end{verbatim}
\subsection{Description}
\verb"Y = freq(Y,frequencies)" returns the mna object with
calculation frequencies set to \verb"frequencies".

%%%%%%%%%%%%%%%%%%%%%%%%%

\vspace{3mm} \hrule
\section{params}\hypertarget{params}
Returns a cell vector containing the circuit elements in the
circuit.
\subsection{Syntax}
\begin{verbatim}
    x = params(Y)
\end{verbatim}
\subsection{Description}
\verb"x = params(Y)" returns a cell vector with the circuit
elements.

\subsection{Examples}
\begin{verbatim}
    >> Y = read_netlist(mna,'test/test.nl')
    ans =
        '+s.*Cpg+1./(s*Lg)'    '-1./(s*Lg)'                 []
        '-1./(s*Lg)'           '+1./(s*Lg)+s.*Cpg'          []
                         []                     []    '+1./Rs'
    ans =
        'Cpg'    'Lg'    'Rs'
    >> params(Y)
    ans =
        'Cpg'    'Lg'    'Rs'
    >>
\end{verbatim}
%%%%%%%%%%%%%%%%%%%%%%%%%

\vspace{3mm} \hrule
\section{calc}\hypertarget{calc}
Calculates a numeric MNA-matrix.
\subsection{Syntax}
\begin{verbatim}
    X = calc(Y,parameters)
\end{verbatim}
\subsection{Description}
\verb"X = calc(Y,parameters)" returns a \verb"xparam"-object
containing the MNA-matrix calculated at the frequencies set by
\verb"freq". \verb"parameters" is a numeric vector containing the
values of the circuit elements. This vector must have the same
ordering as the elements were inserted into the mna-object.

%%%%%%%%%%%%%%%%%%%%%%%%%
\vspace{3mm} \hrule
\section{gnd}\hypertarget{gnd}
Connects specified nodes to ground and thereby reduces the
dimension of the MNA-matrix.
\subsection{Syntax}
\begin{verbatim}
    Y = gnd(Y,conn)
\end{verbatim}
\subsection{Description}
\verb"Y = gnd(Y,conn)", where \verb"conn" is a vector containing
the nodes to be connected to ground.

\subsection{Examples}
Example of a gyrator with nodes 3 and 4 grounded.
\begin{verbatim}
    >> Y=stamp(mna(4),'GY','g',[1 2 3 4])
    ans =
          []    '+g'      []    '-g'
        '-g'      []    '+g'      []
          []    '-g'      []    '+g'
        '+g'      []    '-g'      []
    ans =
        'g'
    >> Y=gnd(Y,[3 4])
    ans =
          []    '+g'
        '-g'      []
    ans =
        'g'
\end{verbatim}

%%%%%%%%%%%%%%%%%%%%%%%%%
\vspace{3mm} \hrule
\section{read\_netlist}\hypertarget{readnetlist}
Reads a netlist and builds a corresponding mna-object.
\subsection{Syntax}
\begin{verbatim}
    Y = read_netlist(Y,file)
\end{verbatim}
\subsection{Description}
\verb"Y = read_netlist(Y,file)", where \verb"file" is a string
containing the path to the netlist file.

The netlist should be a text-file of the following form:
\begin{verbatim}
    C Cpg 1 3
    L Lg  1 2
    C Cpg 2 5
    R Rs  3 4
    GND 3 5
\end{verbatim}
The circuit elements could be given in any order.

\subsection{Examples}
The file \verb"test.nl" contains the netlist above.
\begin{verbatim}
    >> Y = read_netlist(mna,'test/test.nl')
    ans =
        '+s.*Cpg+1./(s*Lg)'    '-1./(s*Lg)'                 []
        '-1./(s*Lg)'           '+1./(s*Lg)+s.*Cpg'          []
                     []                     []    '+1./Rs'
    ans =
        'Cpg'    'Lg'    'Rs'
\end{verbatim}


%%%%%%%%%%%%%%%%%%%%%%%%%
\vspace{3mm} \hrule
\section{display}\hypertarget{display}
Displays the MNA-matrix and a list of circuit variables of a
mna-object.
\subsection{Syntax}
\begin{verbatim}
    display(Y)
\end{verbatim}

%%%%%%%%%%%%%%%%%%%%%%%%%
