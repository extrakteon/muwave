% $Header$
% $Author: fager $
% $Date: 2004-10-21 20:59:19 +0200 (Thu, 21 Oct 2004) $
% $Revision: 223 $
% $Log$
% Revision 1.2  2004/10/21 18:59:06  fager
% Version logging added. Comments from KA implemented.
%
\section{Basic access and manipulation of properties}\label{sec:DACManip}
Once the measurement data has been read into an \verb"meassp"
object, there are several methods implemented to access and
manipulate their contents.

First, it will be described how to access and manipulate the
\verb"measstate" and \verb"measmnt" object properties. The
S-parameter data, which is stored in an \verb"xparam" object is
treated separately in \secref{SPMod}.

\subsection{set and get}
Generally, the two functions \verb"set" and \verb"get" are used to
manipulate and access the properties of an \verb"meassp" object.
It is important to note that these functions can be applied
directly to the \verb"meassp" object, although the property that
is being accessed is in fact member of the \verb"measmnt" or
\verb"measstate" sub-classes. A typical usage of them is shown
below:

\begin{small}
\begin{verbatim}
    >> msp = read_touchstone(meassp,...
        'matlab_milou/test/d0509/d0509_20');
    >> get(msp,'Date')
    ans =
    2002-10-25, 08:50

    >> get(msp,'Ids')
    ans =
        0.0197

    >> get(msp,'Freq')
    ans =
    1.0e+010 *
        0.1000
        0.1490
        ...

    >> msp = set(msp,'Operator','Tintin','Igs',1,...
        'Date',datestr(now))
    Measurement info
        Date : 19-Oct-2004 15:28:00
        Origin : matlab_milou\test\d0509\d0509_20
        Operator : Tintin
        Info :
    Measurement state
        MeasType:   SP
        Vgs:    -0.1
        Vds:    1.5
        Igs:    1
        Ids:    0.019745
        Freq,min:   1E+009
        Freq,max:   5E+010
    Measurement Data
    xparam-object
        type:  S
        reference: 50
        ports: 2
        elements:  101
\end{verbatim}
\end{small}

\subsection{The dot-operator}
A more convenient way of accessing and manipulating \verb"meassp"
data than using \verb"set" and \verb"get" is to use the ``dot''-
operator. The following expressions are equivalent and illustrate
the principle of operation:

\begin{small}
\begin{verbatim}
    >> msp = set(msp,'Operator','Phyllis');
    >> msp.Operator = 'Phyllis';

    >> f = get(msp,'Freq');
    >> f = msp.Freq;
\end{verbatim}
\end{small}

The dot-assignments above automatically call the
\verb"@meassp/subsasgn.m" function, which overloads the built-in
\verb"subsasgn" function that is used to handle the dot-operator
for other input types. See \verb">> help subsasgn" for further
details. Similar to \verb"get", \verb"subsref" plays the same role
when the dot-operator is used to access a parameter value.

\subsection{Adding a property}
It is sometimes required to add a property that was not defined in
the TouchStone file. For example, we might want to add a
\verb"VNA" property that we can use to identify which VNA was used
during the measurements. The procedure to add a
\verb"measmnt"-property is easy:

\begin{small}
\begin{verbatim}
    >> msp = addprop(msp,'VNA','HP 8510C')
    Measurement info
        Date : 2002-10-25, 08:50
        Origin : matlab_milou\test\d0509\d0509_20
        Operator :
        Info :
        VNA : HP 8510C
        ...
\end{verbatim}
\end{small}
To add a \verb"measstate" property, the procedure is a bit
different. First, the \verb"measstate" object must be extracted
from the \verb"meassp" object. The new property can then be added
to the \verb"measstate" object. This modified \verb"measstate"
object should then be put back into the original \verb"meassp"
object. It is maybe more easy to understand in code:

\begin{small}
\begin{verbatim}
    >> mstate = get(msp,'measstate');
    >> mstate = addprop(mstate,'Temp',75);
    >> msp = set(msp,'measstate',mstate)
    ...
    Measurement state
        MeasType:   SP
        Vgs:    -0.1
        Vds:    1.5
        Igs:    -2.784E-007
        Ids:    0.019745
        Freq,min:   1E+009
        Freq,max:   5E+010
        Temp:   75
    ...
\end{verbatim}
\end{small}

\subsection{Reduction and extension of frequency range}
Another operation that is commonly encountered is to remove
undesired frequencies, or to reduce the number of frequency
points. Two methods can be used.

The first case applies when the frequencies remain the same, but
some points should be removed. The \verb"()" indexing method can
then be applied. The following example shows an example of how to
carry out this:

\begin{small}
\begin{verbatim}
    >> f = msp.freq;
    >> freq_index = find(f>1e9 & f<10e9);
    >> msp_reduced = msp(freq_index)
        ...
        Freq,min:   1.49E+009
        Freq,max:   9.82E+009
    Measurement Data
    xparam-object
        type:  S
        reference: 50
        ports: 2
        elements:  18
        ...
\end{verbatim}
\end{small}

If the desired frequencies differ from the ones during
measurement, an interpolation method can be used to find a new
\verb"meassp" object corresponding to the new frequency vector.
The \verb"meassp" function \verb"interp" is designed for this
purpose:

\begin{small}
\begin{verbatim}
    >> new_freq = linspace(1e9,1e10,51);
    >> msp_new = interp(msp,new_freq)
        ...
        Freq,min:   1E+009
        Freq,max:   1E+010
    Measurement Data xparam-object
        type:  S
        reference: 50
        ports: 2
        elements:  51
        ...
\end{verbatim}
\end{small}
