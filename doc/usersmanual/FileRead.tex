% $Header$
% $Author: fager $
% $Date: 2004-10-21 20:59:19 +0200 (Thu, 21 Oct 2004) $
% $Revision: 223 $
% $Log$
% Revision 1.2  2004/10/21 18:59:06  fager
% Version logging added. Comments from KA implemented.
%
\section{Reading and storing single measurement files}
The easiest and best way to introduce the functionality of Milou
is probably by an example. The example measurement used here is
for an InP-HEMT device biased in the active region. The
measurement is stored as a TouchStone file at: \newline
\verb"matlab_milou/test/d0509/d0509_20".

\subsection{The TouchStone file format}
The most commonly used file format for storing S-parameter files
is the TouchStone format\footnote{TouchStone is a registered
trademark of Agilent Corporation. A complete specification of the
file format can be found in either
www.eda.org/pub/ibis/connector/touchstone\_spec11.pdf or the
Agilent ADS manual.}. These files are typically identified by a
.S2P file extension (although not the case for the example file
above), where 2 denotes the number of ports in the measurement
data. Below is shown an example of how the first couple of lines
of a Touchstone file typically look like:
\begin{small}
\begin{verbatim}
    !Date: 2002-10-25, 08:50

    !Vgs = -1.000000E-1
    !Vds = 1.500000E+0
    !Ig = -2.784000E-7
    !Id = 1.974500E-2

    #HZ S RI R 50
    1.000000E+9 9.921265E-1 -6.970215E-2 -3.964478E+0 ...
    1.490000E+9 9.887085E-1 -1.027222E-1 -3.976196E+0 ...
    1.980000E+9 9.827881E-1 -1.362915E-1 -3.920410E+0 ...
    2.470000E+9 9.783020E-1 -1.674194E-1 -3.943237E+0 ...
    2.960000E+9 9.721069E-1 -1.990662E-1 -3.927979E+0 ...
        ...
\end{verbatim}
\end{small}
Rows starting with exclamation marks (!) indicate comments, but
are here used to store additional information about the
measurements. In this case, which is typical, the comments reveal
the biasing conditions as well as the date of measurement.

The next important section is the line starting with \verb"#".
This line reveals the format of the following data. First on that
line, \verb"HZ" is written to indicate that the first column of
the data, which is always the frequency, is given in Hertz.
Thereafter indicates \verb"S" that the data is indeed S-parameter
data. The following \verb"RI" tells us that the S-parameters are
given in ``$\mathrm{Re}(S)~\mathrm{Im}(S)$'' format. Other
alternatives are \verb"MA" and \verb"DB", corresponding to
``$|S|~\angle S$'' and ``$20 log_{10}(|S|)~\angle S$'',
respectively. Finally, \verb"R 50" indicates that the S-parameters
are given for a $50 \Omega$ system impedance. The remainder of the
file consists of lines in the following format:
\begin{small}
\begin{verbatim}
    freq real(S11) imag(S11) real(S21) imag(S21) real(S12)...
\end{verbatim}
\end{small}
where in this case the \verb"RI" format has been assumed.

\subsection{read\_touchstone}
The \verb"read_touchstone" function allows a straightforward way
of importing TouchStone files into Milou:

\begin{small}
\begin{verbatim}
    >> msp = read_touchstone(meassp,...
        'matlab_milou/test/d0509/d0509_20')
    Measurement info
        Date : 2002-10-25, 08:50
        Origin : matlab_milou\test\d0509\d0509_20
        Operator :
        Info :
    Measurement state
        MeasType:   SP
        Vgs:    -0.1
        Vds:    1.5
        Igs:    -2.784E-007
        Ids:    0.019745
        Freq,min:   1E+009
        Freq,max:   5E+010
    Measurement Data
    xparam-object
        type:  S
        reference: 50
        ports: 2
        elements:  101
\end{verbatim}
\end{small}

The variable \verb"msp" is now an instance of an \verb"meassp"
object, as illustrated in \figref{ClassHierarchy}. The parameters
displayed indicate that \verb"read_touchstone" has parsed the
header in the TouchStone file and put the information in the
corresponding sub-objects. The following sections will describe
how to access and manipulate these parameters and properties.

\subsection{write\_touchstone}
Naturally, it is also possible to store a \verb"meassp" object as
a TouchStone file. The syntax is similar as for
\verb"read_touchstone":
\begin{small}
\begin{verbatim}
    >> write_touchstone(msp,'c:\users\nn\temp\datatest.s2p');
\end{verbatim}
\end{small}
