% $Header$
% $Author$
% $Date$
% $Revision$
% $Log$
% Revision 1.2  2004/10/21 18:59:06  fager
% Version logging added. Comments from KA implemented.
%
\section{Installation}%\hypertarget{Installation}
MuWave has been developed using \matlab v.6.5 (Release 13), which
is assumed to be installed on the user's computer. Please check
the Help/About menu to make sure you are running this or a later
version.

The installation of the MuWave-toolbox consists of a few simple
steps:
\begin{enumerate}
    \item Download the latest version from the MuWave web page at \newline http://www.chalmers.se/mc2/EN/laboratories/microwave-electronics/education/graduate-courses/empirical-modeling.
    \newline The downloaded zip-file contains all \matlab m-files and
documentation necessary.
    \item Extract all files to a directory of your choice. In WinZip, make
    sure to have the ''Use folder names'' option checked.
    \item Add MuWave to \matlab's path. This is done from the
    File menu by selecting ''Set Path...''. Click ''Add Folder''
    in the dialog box that appears and browse to select the ''muwave''
    directory, which was created where you extracted the
    zip-file. Please make sure to press the ''Save button''
    before closing the Set Path dialog box.
    \item Normally, MuWave is now ready for action. However, it
    may in rare cases be necessary to execute the following line before
    MuWave is used the first time:
    \begin{verbatim}
        >> clear classes; rehash;
    \end{verbatim}
\end{enumerate}
You can verify that MuWave has been successfully installed by
creating an empty measurement object. This is done by executing:
\verb">> meassp". The following should be displayed at the command
prompt:
\begin{verbatim}
    >> meassp
    Measurement info
        Date : 18-Oct-2004 20:33:57
        Origin :
        Operator :
        Info :
    Measurement state
        MeasType:   SP
    Measurement Data
    xparam-object
        type:  S
        reference: 50
        ports: 2
        elements:  0
\end{verbatim}
