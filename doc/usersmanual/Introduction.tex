% $Header$
% $Author: fager $
% $Date: 2004-10-21 20:59:19 +0200 (Thu, 21 Oct 2004) $
% $Revision: 223 $
% $Log$
% Revision 1.2  2004/10/21 18:59:06  fager
% Version logging added. Comments from KA implemented.
%
\section{Introduction}%\hypertarget{Introduction}
The \emph{Matlab-Milou toolbox}\footnote{Milou is hereafter used
as a short-hand notation for the \matlab-Milou toolbox.} is a
collection of functions to facilitate easy handling of
S-parameters in the \matlab\footnote{\matlab is a trademark of The
Mathorks Inc., Natick,MA, U.S.A} environment. The computational
power and programming framework of \matlab can thereby be applied
to calculations involving measured or simulated data. This makes
it particularly suited for device modelling applications, for
which e.g. CAD tools are of limited use.

The purpose of this document is to describe the basic
functionality offered by the toolbox in order for new users to get
started and acquainted to the most commonly used functions.

\subsection{Further help}
For a more detailed description of each function encountered in
this text, users are referred to the Matlab-Milou Reference manual
and on-line help provided by typing \verb">> help <command name>"
or \verb">> helpwin <command name>" at the \matlab command prompt.
It is, however, assumed that readers of this document are
reasonably familiar with the general \matlab environment.

\subsection{Feedback}
Milou is under continuous development. Most certainly you will
find bugs and annoying inconsistencies as you start using it. It
is therefore of great help for us to get feedback on features that
are missing or if errors are found. Therefore, please do not
hesitate to contact us if you have anything you are wondering
about, report errors you have found, or if you have developed your
own functionality that you want to share.

\vspace{5mm} \setlength{\tabcolsep}{0.5cm} \noindent
\begin{center}
    \begin{tabular}{ll}
    Kristoffer Andersson & Christian Fager\\
    kristoffer.andersson@mc2.chalmers.se &
    christian.fager@mc2.chalmers.se\\
\end{tabular}
\end{center}
