% $Header$
% $Author$
% $Date$
% $Revision$
% $Log$
% Revision 1.4  2005/02/27 18:32:56  fager
% Correction from BH implemented.
%
% Revision 1.3  2005/02/22 07:31:00  fager
% Corrections from M.Kelly implemented
%
% Revision 1.2  2004/10/21 18:59:06  fager
% Version logging added. Comments from KA implemented.
%
\section{X-parameter data access and manipulation}\label{sec:SPMod}
The most valuable feature of MuWave is for most users the
possibility to handle and manipulate S/Y/Z/T\footnote{Hereafter
``X-parameter'' is used to refer either of the
S/Y/Z/T-parameters.} in a convenient way. This section gives an
overview of the variety of possibilities offered.

\subsection{Direct parameter access}
Any of the X-parameters can be directly accessed from the
\verb"meassp" object by using the ``dot-operator'' notation or a
get command:

\begin{small}
\begin{verbatim}
    >> msp.Z11
    ans =
    1.0e+002 *
    1.0467 - 5.6792i
    0.8499 - 3.9010i
    ...

    >> 20*log10(abs(get(msp,'S21')))
    ans =
    11.9806
    12.0251
    11.9318
    ...
\end{verbatim}
\end{small}

By this it was also illustrated how MuWave automatically converts
from the S-para\-meters in \verb"msp" into the desired type, as in
\verb"msp.Z11". Note that these functions all return a row vector
that can be involved in any complex mathematical vector
manipulation that \matlab supports.

\subsection{Parameter assignment}
It is also possible to assign new values for a particular
X-parameter data using a simple assignment:
\begin{small}
\begin{verbatim}
    >> msp.S11 = 0.1*msp.S11;
\end{verbatim}
\end{small}
If the parameter type being assigned differs from the type of the
\verb"meassp" object it is operating on, a conversion takes place
to match the type of parameter that is being assigned:
\begin{small}
\begin{verbatim}
    >> msp
    ...
    xparam-object
    	 type:	S
    	 reference:	50
    	 ports:	2
    	 elements:	101
    	 freq:	1E+009 -> 5E+010
    ...
    >> msp.Z11 = zeros(length(msp.freq),1);
    >> msp
    ...
    xparam-object
    	 type:	Z
    	 reference:	50
    	 ports:	2
    	 elements:	101
    	 freq:	1E+009 -> 5E+010
\end{verbatim}
\end{small}

\subsection{Matrix operations}
There are three levels one can choose from when doing matrix
operations on one or several X-parameter objects. Which to choose
depends on the complexity of the operations involved.

\subsubsection{xparam}
On the highest level it is possible to operate directly with the
\verb"xparam"-objects. The \verb"xparam"-object is extracted from
the \verb"meassp" object using:
\begin{small}
\begin{verbatim}
    xparam-object
    	 type:	S
    	 reference:	50
    	 ports:	2
    	 elements:	101
    	 freq:	1E+009 -> 5E+010
\end{verbatim}
\end{small}
where, again, the dot-operator has been applied to the
\verb"meassp" object.

The \verb"xparam" objects contain information about i.e. the type
of data (S/Z/Y/T). MuWave therefore automatically converts between
these types so that they match i.e. when adding two objects.
\begin{small}
\begin{verbatim}
    >> Zp = msp.Z;
    >> Yp = msp.Y;
    >> Zp_sum = Zp + Yp
    Warning: XPARAM.PLUS: Arguments not of same type. Conversion performed
    > In xparam.plus at 25
    xparam-object
    	 type:	Z
    	 reference:	50
    	 ports:	2
    	 elements:	101
    	 freq:	1E+009 -> 5E+010
\end{verbatim}
\end{small}
Although only shown for addition here, a lot of other algebraic
manipulations have been implemented for the \verb"xparam" class.
However, be careful that the conversions sometimes taking place
when operating on \verb"xparam" objects of different type may give
undesired results. Please check out the online help for details on
how each operation works.

It often turns out to be more reliable to operate on the data
directly by accessing the underlying \verb"arraymatrix" object.

\subsubsection{arraymatrix}
The \verb"arraymatrix" is accessed from the \verb"xparam" object:
\begin{small}
\begin{verbatim}
    >> Sp = msp.S;
    >> Sp_array = get(Sp,'arraymatrix')
    arraymatrix-object
        dimension:  2 x 2
        elements:   101
    First matrix element:
    0.9921 - 0.0697i   0.0009 + 0.0037i
    -3.9645 + 0.2474i   0.8495 - 0.0128i
\end{verbatim}
\end{small}
The \verb"arraymatrix" can also be directly accessed from the
\verb"meassp" object by
\begin{small}
\begin{verbatim}
    >> Sp_array = get(msp,'arraymatrix');
\end{verbatim}
\end{small}

Individual parameters of the \verb"arraymatrix" object are
accessed and assigned using parentheses \verb"()",
\begin{small}
\begin{verbatim}
    >> Sp_array2 = Sp_array;
    >> Sp_array2(1,1) = Sp_array2(2,2)
    arraymatrix-object
        dimension:  2 x 2
        elements:   101
    First matrix element:
    0.8495 - 0.0128i   0.0009 + 0.0037i
    -3.9645 + 0.2474i   0.8495 - 0.0128i
\end{verbatim}
\end{small}

Matrix operations are carried out in a similar straightforward fashion,
\begin{small}
\begin{verbatim}
    >> Sp_array3 = Sp_array2*Sp_array
    arraymatrix-object
        dimension:  2 x 2
        elements:   101
    First matrix element:
    0.8376 - 0.0866i   0.0016 + 0.0063i
    -7.2809 + 0.7830i   0.7172 - 0.0365i
\end{verbatim}
\end{small}
where the matrix multiplication of \verb"Sp_array2" and
\verb"Sp_array" is carried out independently for each frequency
point. A variety of other algebraic operations are implemented for
the \verb"arraymatrix" class.

Finally, it is usually to put the manipulated \verb"arraymatrix"
object back into the \verb"meassp" context:
\begin{small}
\begin{verbatim}
    >> Sp = xparam(Sp_array3,'S',50);
    >> msp_new = msp;
    >> msp_new.data = Sp;
\end{verbatim}
\end{small}

\subsubsection{3d-matrix}
In rare cases, the functions implemented for the
\verb"arraymatrix" class are not sufficient to solve a particular
problem. The way out is then to access the X-parameter data in its
raw form, i.e. in the way it is \emph{actually} stored in \matlab.

The data is stored as a 3-dimensional matrix, where the first two
dimensions correspond to the row and columns of the X-parameter
matrix at each frequency. The third dimension is the frequency.
This means that, e.g. a two-port S-parameter measurement with 101
frequency points will be stored as a 2x2x101-sized complex vector.

This raw data matrix can be accessed directly from the
\verb"xparam", \verb"arraymatrix", or the \verb"meassp" objects by
getting the \verb"mtrx" property:
\begin{small}
\begin{verbatim}
    >> Sp = msp.S;
    >> raw_data = get(Sp,'mtrx')
    raw_data(:,:,1) =
    0.9921 - 0.0697i   0.0009 + 0.0037i
    -3.9645 + 0.2474i   0.8495 - 0.0128i

    raw_data(:,:,2) =
    0.9887 - 0.1027i   0.0011 + 0.0053i
    -3.9762 + 0.3613i   0.8527 - 0.0303i
    ...
    ...
    raw_data(:,:,101) =
    -0.5639 - 0.5608i   0.0775 + 0.0227i
    1.1758 + 1.5074i   0.2195 - 0.7032i
\end{verbatim}
\end{small}
After manipulation, it can be used to create an \verb"arraymatrix"
object,
\begin{small}
\begin{verbatim}
    >> new_array = arraymatrix(raw_data);
\end{verbatim}
\end{small}
which, in turn can be converted back into \verb"xparam" and
\verb"meassp" objects as was described above.
