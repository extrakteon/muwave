% $Header$
% $Author: fager $
% $Date: 2009-04-28 20:51:04 +0200 (ti, 28 apr 2009) $
% $Revision: 99 $
% $Log$
% Revision 1.3  2005/02/22 07:31:00  fager
% Corrections from M.Kelly implemented
%
% Revision 1.2  2004/10/21 18:59:06  fager
% Version logging added. Comments from KA implemented.
%

\section{Sweeps of measurements}\label{sec:sweeps}
MuWave has not only functionality to handle single measurements,
using the \verb"meassp" class, but also a series of measurements.
Typically, this is a series of measurements obtained at different
bias points. The class used to handle these swept measurements is
called \verb"meassweep".

MuWave can create \verb"meassweep" objects from either a directory
with multiple TouchStone files or from an MDIF file\footnote{See
Agilent ADS documentation for details about the MDIF file
format.}.

\subsection{Importing multiple Touchstone files}
Multiple TouchStone files are imported into an \verb"meassweep"
object using the function \verb"read_milousweep". The following
example shows a typical call with this function:
\begin{small}
\begin{verbatim}
    >> sweep = read_milousweep(meassweep,...
    Measurement info
        Date : 14-Apr-2009 10:54:07
        Origin : muwave\test\d0509_all\d0506_*
        Operator :
        Info :
    Sweep info
    	Vgs,min: -0.5
    	Vgs,max: 0.2
    	Vds,min: 0.5
    	Vds,max: 1.5
    	Igs,min: -7.94141e-007
    	Igs,max: 2.05474e-006
    	Ids,min: 0.00343814
    	Ids,max: 0.0297255
    	Number of measurements: 24
\end{verbatim}
\end{small}
Please consult \verb">> help read_milousweep" to find the details
on how to call this function.

\subsection{Importing MDIF-files}
MDIF files allow a convenient way of transferring multi
dimensional data to and from Agilent ADS. The structure of the
MDIF files are described in the Agilent ADS manual and not
repeated here.

The function \verb"read_mdif" can be used to import MDIF data into
a \verb"meassweep" object in Milou. It should be stressed that, at
the time of writing, this function is still under development and
may not operate as expected!

\subsection{Importing Maury ATS swept bias S-parameters}
The function \verb"read_s2b" allows Milou to import swept bias
S-parameter files created by the Maury Microwave ATS software.
Please consult the on-line help for more information about this
function.

\subsection{Accessing and indexing meassweep objects}
A \verb"meassweep" object can thought of as a vector of
\verb"meassp" objects. An individual \verb"meassweep" object, or a
subset of the \verb"meassweep" can be accessed using the
parenthesis \verb"()" operator,
\begin{small}
\begin{verbatim}
    >> sweep(1)
    Measurement info
        Date : 2002-10-24, 19:23
        Origin : muwave\test\d0509_all\d0506_0
        Operator :
        Info :
    Measurement state
    	MeasType:	SP
    	Vgs:	-0.5
    	Vds:	0.5
    	Igs:	-6.43067e-007
    	Ids:	0.00343814
    xparam-object
    	 type:	S
    	 reference:	50
    	 ports:	2
    	 elements:	101
    	 freq:	1E+009 -> 5E+010
    >> sweep(1:10)
    Measurement info
        Date : 14-Apr-2009 10:54:07
        Origin : muwave\test\d0509_all\d0506_*
        Operator :
        Info :
    Sweep info
    	Vgs,min: -0.5
    	Vgs,max: 0.2
    	Vds,min: 0.5
    	Vds,max: 1
    	Igs,min: -6.97489e-007
    	Igs,max: 7.27892e-007
    	Ids,min: 0.00343814
    	Ids,max: 0.0233942
    	Number of measurements: 10
\end{verbatim}
\end{small}

\subsection{The dot-operator}
An individual parameter of the sweep objects can be accessed using
the dot-operator. The following example shows how to access the
\verb"Vgs"-parameter and also how this can be used to use reduce
the sweep-object to contain only bias points with \verb"Vgs" $=0$.
\begin{small}
\begin{verbatim}
    >> sweep.Vgs
    ans =
    Columns 1 through 7
    -0.5000   -0.4000   -0.3000   -0.2000   -0.1000   -0.0000    0.1000
    Columns 8 through 14
        0.2000   -0.5000   -0.4000   -0.3000   -0.2000   -0.1000   -0.0000
    ...

    >> bias_index = find(abs(sweep.Vgs)<0.01);
    >> sweep(bias_index)
    Measurement info
        Date : 14-Apr-2009 10:54:07
        Origin : muwave\test\d0509_all\d0506_*
        Operator :
        Info :
    Sweep info
    	Vgs: -2.77556e-017
    	Vds,min: 0.5
    	Vds,max: 1.5
    	Igs,min: -1.09436e-007
    	Igs,max: -4.34143e-008
    	Ids,min: 0.0212195
    	Ids,max: 0.0257597
    	Number of measurements: 3
\end{verbatim}
\end{small}

\subsection{Adding measurements}
Single measurements can be added to the \verb"meassweep" object
with the \verb"add" function,

\begin{small}
\begin{verbatim}
    >> msp = read_touchstone(meassp,'muwave\test\d0509\d0509_20');
    >> sweep_total = add(sweep,msp);
    >> length(sweep_total)-length(sweep)
    ans =
        1
\end{verbatim}
\end{small}
The same function, \verb"add", can also be used to merge two
\verb"meassweep" objects.

\subsection{Writing swept data}
For the moment, no functions have been implemented to write a
\verb"meassweep" object to multiple TouchStone files or into an
MDIF-file.
